\PassOptionsToPackage{unicode=true}{hyperref} % options for packages loaded elsewhere
\PassOptionsToPackage{hyphens}{url}
%
\documentclass[]{article}
\usepackage{lmodern}
\usepackage{amssymb,amsmath}

% settings
\usepackage{minted}
%



\usepackage{ifxetex,ifluatex}
\usepackage{fixltx2e} % provides \textsubscript
\ifnum 0\ifxetex 1\fi\ifluatex 1\fi=0 % if pdftex
  \usepackage[T1]{fontenc}
  \usepackage[utf8]{inputenc}
  \usepackage{textcomp} % provides euro and other symbols
\else % if luatex or xelatex
  \usepackage{unicode-math}
  \defaultfontfeatures{Ligatures=TeX,Scale=MatchLowercase}
    \setmainfont[]{Source Han Serif CN}
    \setsansfont[]{Source Han Sans CN}
    \setmonofont[Mapping=tex-ansi]{Source Code Pro}
  \ifxetex
    \usepackage{xeCJK}
    \setCJKmainfont[]{Source Han Serif CN}
  \fi
  \ifluatex
    \usepackage[]{luatexja-fontspec}
    \setmainjfont[]{Source Han Serif CN}
  \fi
\fi
% use upquote if available, for straight quotes in verbatim environments
\IfFileExists{upquote.sty}{\usepackage{upquote}}{}
% use microtype if available
\IfFileExists{microtype.sty}{%
\usepackage[]{microtype}
\UseMicrotypeSet[protrusion]{basicmath} % disable protrusion for tt fonts
}{}
\IfFileExists{parskip.sty}{%
\usepackage{parskip}
}{% else
\setlength{\parindent}{0pt}
\setlength{\parskip}{6pt plus 2pt minus 1pt}
}
\usepackage{hyperref}
\hypersetup{
            pdfborder={0 0 0},
            breaklinks=true}
\urlstyle{same}  % don't use monospace font for urls
\usepackage[margin=2cm]{geometry}
\setlength{\emergencystretch}{3em}  % prevent overfull lines
\providecommand{\tightlist}{%
  \setlength{\itemsep}{0pt}\setlength{\parskip}{0pt}}
\setcounter{secnumdepth}{0}
% Redefines (sub)paragraphs to behave more like sections
\ifx\paragraph\undefined\else
\let\oldparagraph\paragraph
\renewcommand{\paragraph}[1]{\oldparagraph{#1}\mbox{}}
\fi
\ifx\subparagraph\undefined\else
\let\oldsubparagraph\subparagraph
\renewcommand{\subparagraph}[1]{\oldsubparagraph{#1}\mbox{}}
\fi


% set default figure placement to htbp
\makeatletter
\def\fps@figure{htbp}
\makeatother

\usepackage{minted}



\date{}

\title{\vspace{50mm} \huge Code Template \\[20pt]}
\author{There is no gay @ CCNUACM \\[10pt] Central China Normal University}
\date{\today}


\begin{document}

\begin{titlepage}

\maketitle

\end{titlepage}

\newpage

\renewcommand\labelitemi{$\bullet$}

{
\setcounter{tocdepth}{3}
\tableofcontents
\newpage
}








\hypertarget{ux6742ux9879}{%
\section{0-杂项}\label{ux6742ux9879}}

\begin{center}\rule{0.5\linewidth}{0.5pt}\end{center}

\hypertarget{ux5febux8bfb}{%
\subsection{快读}\label{ux5febux8bfb}}

\begin{minted}[fontsize=\footnotesize,breaklines,linenos]{cpp}
inline int read()
{
    int x = 0, w = 1;
    char ch = 0;
    while (ch < '0' || ch > '9')
    {
        ch = getchar();
        if (ch == '-')
        {
            w = -1;
        }
    }
    while (ch >= '0' && ch <= '9')
    {
        x = x * 10 + ch - '0';
        ch = getchar();
    }
    return x * w;
}
\end{minted}

\hypertarget{judge.sh}{%
\subsection{\texorpdfstring{\texttt{judge.sh}}{judge.sh}}\label{judge.sh}}

Assume using directory ``./contest''.

\begin{verbatim}
../contest:
a.cpp  b.cpp  samples-a  samples-b  judge.sh

../contest/samples-a:
1.ans  1.in

../contest/samples-b:
1.ans  1.in  2.ans  2.in
\end{verbatim}

\texttt{judge.sh}

\begin{minted}[fontsize=\footnotesize,breaklines,linenos]{bash}
#!bin/bash
set -e
[ $# == 2 ] || { echo invalid args ; exit 1 ; }
g++ $2.cpp || { echo CE ; exit 1 ; }
src=./samples-$1
dir=$1-test
mkdir -p $dir
cp $src/* $dir/ 
cd $dir
mv ../a.out ./$2
for input in *.in; do
  [ $input == "*.in" ] && exit 0
  cas=${input%.in}
  output=$cas.out
  answer=$cas.ans
  timeout 1 ./$2 < $input > $output 2> $cas.err || { echo Case $cas : TLE or RE ; continue ; }
  if diff -Za $output $answer > $cas.dif ; then
    echo Case $cas : AC
  else
    echo Case $cas : WA
    cat $cas.dif $cas.err
  fi
done
\end{minted}

command:

\begin{minted}[fontsize=\footnotesize,breaklines,linenos]{bash}
cd ./contest
bash judge.sh a a.cpp
\end{minted}

\hypertarget{ux5fc3ux6001ux5d29ux4e86}{%
\subsection{心态崩了}\label{ux5fc3ux6001ux5d29ux4e86}}

\begin{itemize}
\tightlist
\item
  \texttt{(int)v.size()}
\item
  \texttt{1LL\ \textless{}\textless{}\ k}
\item
  递归函数用全局或者 static 变量要小心
\item
  预处理组合数注意上限
\item
  想清楚到底是要 \texttt{multiset} 还是 \texttt{set}
\item
  提交之前看一下数据范围,测一下边界
\item
  数据结构注意数组大小 (2倍,4倍)
\item
  字符串注意字符集
\item
  如果函数中使用了默认参数的话,注意调用时的参数个数。
\item
  注意要读完
\item
  构造参数无法使用自己
\item
  树链剖分/dfs 序,初始化或者询问不要忘记 idx, ridx
\item
  排序时注意结构体的所有属性是不是考虑了
\item
  不要把 while 写成 if
\item
  不要把 int 开成 char
\item
  清零的时候全部用 0\textasciitilde n+1。
\item
  模意义下不要用除法
\item
  哈希不要自然溢出
\item
  最短路不要 SPFA,乖乖写 Dijkstra
\item
  上取整以及 GCD 小心负数
\item
  mid 用 \texttt{l\ +\ (r\ -\ l)\ /\ 2} 可以避免溢出和负数的问题
\item
  小心模板自带的意料之外的隐式类型转换
\item
  求最优解时不要忘记更新当前最优解
\item
  图论问题一定要注意图不连通的问题
\item
  处理强制在线的时候 lastans 负数也要记得矫正
\item
  不要觉得编译器什么都能优化
\item
  分块一定要特判在同一块中的情况
\end{itemize}

\hypertarget{sol.cpp}{%
\subsection{sol.cpp}\label{sol.cpp}}

\begin{minted}[fontsize=\footnotesize,breaklines,linenos]{cpp}
#include <bits/stdc++.h>
using namespace std;
using ll = long long;
using VI = vector<ll>;
using PII = pair<int, int>;
template <typename T> using fc = function<T>;
using Graph = vector<vector<int>>;
#define pb push_back
#define debug(c) cerr << #c << " = " << c << endl;
#define rg(x) x.begin(), x.end()
#define rep(a, b, c) for (auto a = (b); (a) < (c); a++)
#define repe(a, b, c) for (auto a = (b); (a) <= (c); a++)
const int MOD = 998244353;
const int N = 0;
#ifdef ONLINE_JUDGE
#define cerr assert(false);
#endif

void solve()
{
    
}

int main()
{
    std::ios::sync_with_stdio(false);
    cin.tie(nullptr);
    int T = 1;
    cin >> T;
    while (T--)
        solve();

    return 0;
}
\end{minted}

\hypertarget{ux4e09ux5206}{%
\subsection{三分}\label{ux4e09ux5206}}

\begin{minted}[fontsize=\footnotesize,breaklines,linenos]{cpp}
double cal()
{
    double l = 0, r = 1e10;
    for (int i = 1; i <= 100; i++)
    {
        double m1 = l + (r - l) / 3;
        double m2 = (r - l) / 3 * 2 + l;
        if (f(m1) < f(m2))
            l = m1;
        else
            r = m2;
    } // 求最大值
    return f(l);
}

int cal()
{
    int l = 0, r = 1e10;
    while (l + 2 < r)
    {
        int m1 = l + (r - l) / 3;
        int m2 = (r - l) / 3 * 2 + l;
        if (f(m1) < f(m2))
        {
            l = m1;
        }
        else
        {
            r = m2;
        }
    }
    int ans = f(l);
    for (int i = l + 1; i <= r; i++)
    {
        ans = max(ans, f(i));
    }
}
\end{minted}

\hypertarget{ux5b57ux7b26ux4e32}{%
\section{1-字符串}\label{ux5b57ux7b26ux4e32}}

\begin{center}\rule{0.5\linewidth}{0.5pt}\end{center}

\hypertarget{kmp}{%
\subsection{KMP}\label{kmp}}

\begin{minted}[fontsize=\footnotesize,breaklines,linenos]{cpp}
int f[N];
void kmp(string s, string p)
{
    p += '@';
    p += s;
    for (int i = 1; i < p.size(); i++)
    {
        int j = f[i - 1];
        while (j && p[j] != p[i])
            j = f[j - 1];
        if (p[j] == p[i])
            f[i] = j + 1;
    }
}
\end{minted}

\hypertarget{manacher}{%
\subsection{Manacher}\label{manacher}}

\begin{minted}[fontsize=\footnotesize,breaklines,linenos]{cpp}
#include <cstdio>
using namespace std;
using ll = long long;
// !!! N = n * 2, because you need to insert '#' !!!
const int N = 3e7;
#define min(A, B) ((A > B) ? B : A)
// p[i]: range of the palindrome i-centered. 
int p[N];
// s: the string.
char s[N] = "@#";
// l: length of s.
int l = 2;

int main()
{
    char tmp = getchar();
    while (tmp > 'z' || tmp < 'a')
        tmp = getchar();
    while (tmp <= 'z' && tmp >= 'a')
        s[l++] = tmp, s[l++] = '#', tmp = getchar();
    /*<--- input & preparation --->*/
    int m = 0, r = 0;
    ll ans = 0;
    for (int i = 1; i < l; i++)
    {
        // evaluate p[i]
        if (i <= r)
            p[i] = min(p[m * 2 - i], r - i + 1);
        else
            p[i] = 1;
        // brute force!
        while (s[i - p[i]] == s[i + p[i]])
            ++p[i];
        // maintain m, r
        if (i + p[i] > r)
        {
            r = i + p[i] - 1;
            m = i;
        }
        // find the longest p[i]
        if (p[i] > ans)
            ans = p[i]; 
    }
    printf("%lld", ans - 1);

    return 0;
}
    
\end{minted}

\hypertarget{hash}{%
\subsection{hash}\label{hash}}

\hypertarget{ux968fux673aux7d20ux6570ux8868}{%
\subsubsection{随机素数表}\label{ux968fux673aux7d20ux6570ux8868}}

42737, 46411, 50101, 52627, 54577, 191677, 194869, 210407, 221831,
241337, 578603, 625409, 713569, 788813, 862481, 2174729, 2326673,
2688877, 2779417, 3133583, 4489747, 6697841, 6791471, 6878533, 7883129,
9124553, 10415371, 11134633, 12214801, 15589333, 17148757, 17997457,
20278487, 27256133, 28678757, 38206199, 41337119, 47422547, 48543479,
52834961, 76993291, 85852231, 95217823, 108755593, 132972461, 171863609,
173629837, 176939899, 207808351, 227218703, 306112619, 311809637,
322711981, 330806107, 345593317, 345887293, 362838523, 373523729,
394207349, 409580177, 437359931, 483577261, 490845269, 512059357,
534387017, 698987533, 764016151, 906097321, 914067307, 954169327

1572869, 3145739, 6291469, 12582917, 25165843, 50331653
(适合哈希的素数)

\begin{minted}[fontsize=\footnotesize,breaklines,linenos]{cpp}
#include <string>
using ull = unsigned long long;
using std::string;
const int mod = 998244353;

int n;
ull Hash[2000200]; // 自然溢出法用unsigned类型
ull RHash[2000200];
ull base[2000200];
void init()
{
    base[0] = 1;
    for (int i = 1; i <= 2000010; i++)
    {
        base[i] = base[i - 1] * 131 % mod;
    }
}
void get_hash(string s)
{
    for (int i = 1; i <= (int)s.size(); i++)
    {
        Hash[i] = Hash[i - 1] * base[1] % mod + s[i - 1];
        Hash[i] %= mod;
    }
}
void get_Rhash(string s)
{
    for (int i = (int)s.size(); i >= 1; i--)
    {
        RHash[s.size() - i + 1] = RHash[s.size() - i] * base[1] % mod + s[i - 1];
        RHash[i] %= mod;
    }
}
ull getR(int l, int r)
{
    if (l > r)
        return 0;
    return (RHash[r] - (RHash[l - 1] * base[r - l + 1]) % mod + mod) % mod;
}
ull get(int l, int r)
{
    if (l > r)
        return 0;
    return (Hash[r] - (Hash[l - 1] * base[r - l + 1]) % mod + mod) % mod;
}
\end{minted}

\hypertarget{ux5b57ux5178ux6811}{%
\subsection{字典树}\label{ux5b57ux5178ux6811}}

\begin{minted}[fontsize=\footnotesize,breaklines,linenos]{cpp}
#include <string>
using std::string;
/* Last modified: 23/07/03 */
// Trie for string and prefix
class Trie
{

    static const int trie_tot_size = 1e5;
    // trie_node_size: modify if get() is modified.
    static const int trie_node_size = 64;
    int tot = 0;
    // end: reserved for count
    const int end = 63;
    int (*nxt)[trie_node_size];

  public:
    Trie()
    {
        nxt = new (int[trie_tot_size][trie_node_size]);
    }
    int get(char x)
    {
        // modify if x is in certain range, assuming 0-9 or a-z.
        if (x >= 'A' && x <= 'Z')
            return x - 'A';
        else if (x >= 'a' && x <= 'z')
            return x - 'a' + 26;
        else
            return x - '0' + 52;
    }
    int find(string s)
    {
        int cnt = 0;
        for (auto i : s)
        {
            cnt = nxt[cnt][get(i)];
            if (!cnt)
                return 0;
        }
        return cnt;
    }
    void insert(string s)
    {
        int cnt = 0;
        for (auto i : s)
        {
            auto j = get(i);
            // count how many strings went by
            nxt[cnt][end]++;
            if (nxt[cnt][j] > 0)
                // character i already exists.
                cnt = nxt[cnt][j];
            else
            {
                // doesn't exist, new node.
                nxt[cnt][j] = ++tot;
                cnt = tot;
            }
        }
        nxt[cnt][end]++;
    }
    int count(string s)
    {
        int cnt = find(s);
        if (!cnt)
            return 0;
        return nxt[cnt][end];
    }
    void clear()
    {
        for (int i = 0; i <= tot; i++)
            for (int j = 0; j <= end; j++)
                nxt[i][j] = 0;
        tot = 0;
    }
};
\end{minted}

\hypertarget{acux81eaux52a8ux673a}{%
\subsection{AC自动机}\label{acux81eaux52a8ux673a}}

\begin{minted}[fontsize=\footnotesize,breaklines,linenos]{cpp}
#include <queue>
#include <string.h>
#include <string>
using std::string, std::queue;

struct AC_automaton
{
    static const int _N = 1e6;

    int (*trie)[27];
    int tot = 0;
    int *fail;
    int *e;
    AC_automaton()
    {
        trie = new int[_N][27];
        fail = new int[_N];
        e = new int[_N];
        memset(trie, 0, sizeof(trie));
        memset(fail, 0, sizeof(fail));
        memset(e, 0, sizeof(e));
    }

    void insert(string s)
    {
        int now = 0;
        for (auto i : s)
            if (trie[now][i - 'a'])
                now = trie[now][i - 'a'];
            else
                trie[now][i - 'a'] = ++tot, now = tot;
        e[now]++;
    }

    void build()
    {
        queue<int> q;
        for (int i = 0; i < 26; i++)
            if (trie[0][i])
                q.push(trie[0][i]);
        while (q.size())
        {
            int u = q.front();
            q.pop();
            for (int i = 0; i < 26; i++)
                if (trie[u][i])
                    fail[trie[u][i]] = trie[fail[u]][i], q.push(trie[u][i]);
                else
                    trie[u][i] = trie[fail[u]][i];
        }
    }

    int query(string t)
    {
        int u = 0, res = 0;
        for (auto c : t)
        {
            u = trie[u][c - 'a'];
            for (int j = u; j && e[j] != -1; j = fail[j])
                res += e[j], e[j] = -1;
        }
        return res;
    }
};

\end{minted}

\hypertarget{exkmp-z-function}{%
\subsection{exKMP (Z Function)}\label{exkmp-z-function}}

对\texttt{KMP}中\texttt{next}数组的定义稍作改变,就会得到\texttt{exKMP}。

在\texttt{exKMP}中,我们记录\texttt{z{[}i{]}}为:\texttt{s{[}i:{]}} 与
\texttt{s} 的最长公共前缀。

\begin{minted}[fontsize=\footnotesize,breaklines,linenos]{cpp}
/// @brief Z function, or exKMP.
/// @param s the string.
/// @return the Z function array.
vector<int> Z_function(string &s, VI &z)
{
    // indexing from 1
    int len = s.size();
    s = "#" + s;
    // vector<int> z(len + 1);
    z[1] = 0;
    int l = 1, r = 1;
    for (int i = 2; i <= len; i++)
    {
        if (i <= r && z[i - l + 1] < r - i + 1)
            z[i] = z[i - l + 1];
        else
        {
            z[i] = std::max(0, r - i + 1);
            while (i + z[i] <= len && s[z[i] + 1] == s[i + z[i]])
                z[i]++;
        }
        if (i + z[i] - 1 > r)
        {
            l = i;
            r = i + z[i] - 1;
        }
    }
    z[1] = len;
    return z;
}
\end{minted}

\hypertarget{ux6570ux636eux7ed3ux6784}{%
\section{2-数据结构}\label{ux6570ux636eux7ed3ux6784}}

\begin{center}\rule{0.5\linewidth}{0.5pt}\end{center}

\hypertarget{ux6811ux72b6ux6570ux7ec4}{%
\subsection{树状数组}\label{ux6811ux72b6ux6570ux7ec4}}

\begin{minted}[fontsize=\footnotesize,breaklines,linenos]{cpp}
class BIT
{
    int n = 2e6;
    long long *a;

  public:
    BIT(int size) : n(size)
    {
        a = new long long[size + 10];
    }
    void update(int p, long long x)
    {
        while (p <= n)
            a[p] += x, p += (p & (-p));
    }

    long long query(int l, int r)
    {
        long long ret = 0;
        l--;
        while (r > 0)
            ret += a[r], r -= (r & (-r));
        while (l > 0)
            ret -= a[l], l -= (l & (-l));
        return ret;
    }
};

\end{minted}

\begin{minted}[fontsize=\footnotesize,breaklines,linenos]{cpp}
template<typename T>
struct Fenwick{
    int n;
    vector<T> tr;
 
    Fenwick(int n) : n(n), tr(n + 1, 0){}
 
    int lowbit(int x){
        return x & -x;
    }
 
    void modify(int x, T c){//单点添加
        for(int i = x; i <= n; i += lowbit(i)) tr[i] += c;
    }
 
    void modify(int l, int r, T c){//区间添加
        modify(l, c);
        if (r + 1 <= n) modify(r + 1, -c);
    }
 
    T query(int x){
        T res = T();
        for(int i = x; i; i -= lowbit(i)) res += tr[i];
        return res;
    }
 
    T query(int l, int r){
        return query(r) - query(l - 1);
    }
 
    int find_first(T sum){//和出现的第一位置
        int ans = 0; T val = 0;
        for(int i = __lg(n); i >= 0; i--){
            if ((ans | (1 << i)) <= n && val + tr[ans | (1 << i)] < sum){
                ans |= 1 << i;
                val += tr[ans];
            }
        }
        return ans + 1;
    }
 
    int find_last(T sum){
        int ans = 0; T val = 0;
        for(int i = __lg(n); i >= 0; i--){
            if ((ans | (1 << i)) <= n && val + tr[ans | (1 << i)] <= sum){
                ans |= 1 << i;
                val += tr[ans];
            }
        }
        return ans;
    }
 
};
using BIT = Fenwick<int>; 
\end{minted}

\hypertarget{ux5e76ux67e5ux96c6}{%
\subsection{并查集}\label{ux5e76ux67e5ux96c6}}

\begin{minted}[fontsize=\footnotesize,breaklines,linenos]{cpp}
#include <cassert>
#include <iostream>
#include <set>
/* Last modified: 23/08/01 */
class DSU
{
  private:
    int *f;
    int size;

  public:
    DSU(int size) : size(size)
    {
        assert(size > 1);
        f = new int[size + 10];
        for (int i = 1; i <= size; i++)
            f[i] = i;
    }
    int find(int x)
    {
        return f[x] == x ? x : (f[x] = find(f[x]));
    };
    bool same(int x, int y)
    {
        return find(x) == find(y);
    };
    bool merge(int x, int y)
    {
        int fx = find(x), fy = find(y);
        return ((fx != fy) ? f[fx] = fy : false);
    };
    int count()
    {
        std::set<int> s;
        for (int i = 1; i <= size; i++)
            s.insert(find(i));
        return s.size();
    }
};
\end{minted}

\hypertarget{stux8868}{%
\subsection{ST表}\label{stux8868}}

\begin{minted}[fontsize=\footnotesize,breaklines,linenos]{cpp}
// log2(x) 的预处理
// 1. 递推
lg[2] = 1;
for (int i = 3; i < N; i++)
    lg[i] = lg[i / 2] + 1;
// 2. 基于编译期计算
using std::array;
// WARNING: LOG_SIZE may cause CE if too big.
const int LOG_SIZE = 1e5 + 10;
constexpr array<int, LOG_SIZE> LOG = []() {
    array<int, LOG_SIZE> l{0, 0, 1};
    for (int i = 3; i < LOG_SIZE; i++)
        l[i] = l[i / 2] + 1;
    return l;
}();
// 3. 直接计算
int lg(int x)
{
    return 31 - __builtin_clz(x);
}
// STL 提供了 std::lg(), 底数是e.
\end{minted}

\begin{minted}[fontsize=\footnotesize,breaklines,linenos]{cpp}
class SparseTable
{
  private:
    // SIZE depends on range of f[i][0].
    // 22 is suitable for 1e5.
    static const int SIZE = 22;
    // f[i][j] maintains the result from i to i + 2 ^ j - 1;
    int (*f)[SIZE];
    using func = std::function<int(int, int)>;
    func op;
    // length of f from 1 to l;
    int l;

  public:
    SparseTable(int a[][SIZE], func foo, int len) : f(a), op(foo), l(len)
    {
        for (int j = 1; j < SIZE; j++)
            for (int i = 1; i + (1 << j) - 1 <= len; i++)
                // f[i][j] comes from f[i][j - 1].
                // f[i][j - 1], f[i + 2^(j - 1)] cover the range of f[i][j].
                f[i][j] = foo(f[i][j - 1], f[i + (1 << (j - 1))][j - 1]);
    };
    int query(int x, int y)
    {
        int s = LOG[y - x + 1];
        return op(f[x][s], f[y - (1 << s) + 1][s]);
    }
};
\end{minted}

\hypertarget{ux5e26ux61d2ux6807ux8bb0ux7ebfux6bb5ux6811}{%
\subsection{带懒标记线段树}\label{ux5e26ux61d2ux6807ux8bb0ux7ebfux6bb5ux6811}}

\begin{minted}[fontsize=\footnotesize,breaklines,linenos]{cpp}
const int N = 1e6;
int a[N];
int tag[4 * N];
int tree[4 * N];
int n;
void push_up(int p)
{
    tree[p] = tree[ls(p)] + tree[rs(p)];
}
void build(int p, int l, int r)
{
    if (l == r)
    {
        tree[p] = a[l];
        return;
    }
    int mid = (l + r) >> 1;
    build(ls(p), l, mid);
    build(rs(p), mid + 1, r);
    push_up(p);
}
void push_down(int p, int l, int r)
{
    int mid = (l + r) >> 1;
    tag[ls(p)] += tag[p];
    tag[rs(p)] += tag[p];
    tree[ls(p)] += tag[p] * (mid - l + 1);
    tree[rs(p)] += tag[p] * (r - mid);
    tag[p] = 0;
}
void update(int nl, int nr, int k, int p = 1, int l = 1, int r = n)
{
    if (nl <= l && r <= nr)
    {
        tag[p] += k;
        tree[p] += k * (r - l + 1);
        return;
    }
    push_down(p, l, r);
    int mid = (l + r) >> 1;
    if (nl <= mid)
        update(nl, nr, k, ls(p), l, mid);
    if (nr > mid)
        update(nl, nr, k, rs(p), mid + 1, r);
    push_up(p);
}
int query(int x, int y, int l = 1, int r = n, int p = 1)
{
    int res = 0;
    if (x <= l && y >= r)
        return tree[p];
    int mid = (l + r) >> 1;
    push_down(p, l, r);
    if (x <= mid)
        res += query(x, y, l, mid, ls(p));
    if (y > mid)
        res += query(x, y, mid + 1, r, rs(p));
    return res;
}
int main()
{
    int q;
    cin >> n >> q;
    for (int i = 1; i <= n; i++)
        cin >> a[i];
    build(1, 1, n);
    while (q--)
    {
        int op, x, y, k;
        cin >> op;
        if (op == 1)
        {
            cin >> x >> y >> k;
            update(x, y, k);
        }
        else
        {
            cin >> x >> y;
            cout << query(x, y) << endl;
        }
    }
    return 0;
}
\end{minted}

\hypertarget{ux533aux95f4ux6700ux503cux7ebfux6bb5ux6811}{%
\subsection{区间最值线段树}\label{ux533aux95f4ux6700ux503cux7ebfux6bb5ux6811}}

\begin{minted}[fontsize=\footnotesize,breaklines,linenos]{cpp}
struct node
{
    int maxn;
} tree[800005];
int n;
int tag[800005];
void push_down(int p, int l, int r)
{ // 标记下压
    int mid = (r + l) / 2;
    tree[2 * p].maxn += tag[p];
    tree[2 * p + 1].maxn += tag[p];
    tag[2 * p] += tag[p];
    tag[2 * p + 1] += tag[p];
    tag[p] = 0;
}
void update(int l, int r, int k, int cl = 1, int cr = n, int p = 1)
{ // 更新
    if (cl > r || cr < l)
    {
        return;
    }
    if (cl >= l && cr <= r)
    {
        tree[p].maxn += k;
        if (cl < cr)
        {
            tag[p] += k;
        }
    }
    else
    {
        int mid = (cl + cr) >> 1;
        push_down(p, cl, cr);
        if (l <= mid)
            update(l, r, k, cl, mid, 2 * p);
        if (r > mid)
            update(l, r, k, mid + 1, cr, 2 * p + 1);
        tree[p].maxn = max(tree[p << 1].maxn, tree[p * 2 + 1].maxn);
    }
}
int query(int l, int r, int cl = 1, int cr = n, int p = 1)
{ // 查询
    if (cl >= l && cr <= r)
    {
        return tree[p].maxn;
    }
    else
    {
        int mid = (cl + cr) >> 1;
        push_down(p, cl, cr);
        int tmp = 0;
        if (l <= mid)
            tmp = max(tmp, query(l, r, cl, mid, 2 * p));
        if (r > mid)
            tmp = max(tmp, query(l, r, mid + 1, cr, 2 * p + 1));
        return tmp;
    }
}
\end{minted}

\hypertarget{ux5355ux8c03ux961fux5217}{%
\subsection{单调队列}\label{ux5355ux8c03ux961fux5217}}

\begin{minted}[fontsize=\footnotesize,breaklines,linenos]{cpp}
int p[N];
int head=1,tail=0;
for(int i=1;i<=n;i++){
    if(head<=tail&&p[head]==i-k){//当前0          区间长度大于k时扔掉头部
        head++;
    }
    while(head<=tail&&a[p[tail]]<=a[i]) tail--;//此时求最大值
    p[++tail]=i;
    //则head记录区间内最值
}
\end{minted}

\hypertarget{ux52a8ux6001ux89c4ux5212}{%
\section{3-动态规划}\label{ux52a8ux6001ux89c4ux5212}}

\begin{center}\rule{0.5\linewidth}{0.5pt}\end{center}

\hypertarget{ux80ccux5305dp}{%
\subsection{背包DP}\label{ux80ccux5305dp}}

\begin{minted}[fontsize=\footnotesize,breaklines,linenos]{cpp}
// 多重背包
for (int i = 1; i <= n; i++)
{
    int q = a[i].m;
    for (int j = 1; q; j *= 2)
    {
        if (j > q)
        {
            j = q;
        }
        q -= j;
        for (int k = w; k >= j * a[i].w; k--)
        {
            f[k] = max(f[k], f[k - j * a[i].w] + j * a[i].v);
        }
    }
}
\end{minted}

\hypertarget{ux4e8cux8fdbux5236ux5206ux7ec4ux4f18ux5316}{%
\subsubsection{二进制分组优化}\label{ux4e8cux8fdbux5236ux5206ux7ec4ux4f18ux5316}}

\begin{minted}[fontsize=\footnotesize,breaklines,linenos]{cpp}
index = 0;
for (int i = 1; i <= m; i++)
{
    int c = 1, p, h, k;
    cin >> p >> h >> k;
    while (k > c)
    {
        k -= c;
        list[++index].w = c * p;
        list[index].v = c * h;
        c *= 2;
    }
    list[++index].w = p * k;
    list[index].v = h * k;
}
\end{minted}

\hypertarget{ux72b6ux6001ux538bux7f29dp}{%
\subsection{状态压缩DP}\label{ux72b6ux6001ux538bux7f29dp}}

\begin{minted}[fontsize=\footnotesize,breaklines,linenos]{cpp}
int cnt[1024];
int dp[40][1024][90];
int can[2000], num = 0;
int S = 1 << n;
for (int s = 0; s < S; s++)
{
    if ((s << 1) & s)
    {
        continue;
    }
    can[++num] = s;
    for (int j = 0; j < n; j++)
    {
        if ((s >> j) & 1)
        {
            cnt[num]++;
        }
    }
    dp[1][num][cnt[num]] = 1;
}
for (int i = 2; i <= n; i++)
{
    for (int j = 1; j <= num; j++)
    {
        int x = can[j];
        for (int p = 1; p <= num; p++)
        {
            int y = can[p];
            if ((y & x) || ((y << 1) & x) || ((y >> 1) & x))
                continue;
            for (int l = 0; l <= k; l++)
            {
                dp[i][j][cnt[j] + l] += dp[i - 1][p][l];
            }
        }
    }
}
\end{minted}

枚举s的二进制真子集

\begin{minted}[fontsize=\footnotesize,breaklines,linenos]{cpp}
for(int j=s;j;j=(j-1)&s){
    //...
}
\end{minted}

\hypertarget{ux6570ux4f4ddp}{%
\subsection{数位DP}\label{ux6570ux4f4ddp}}

现在问有多少的数比12345小

1.关于前导0:00999→999,09999→9999

2.关于limit前面的数是否紧贴上限

如果前面的数是紧贴上限的,当前这位枚举的上限便是当前数的上限

如果前面的数不是紧贴上限的,当前这位枚举的上限便是 9

3.关于DP维度

一般来说,DFS有几个状态,DP就几个维度  

比如现在DP就是DP {[}pos{]} {[}limt{]} {[}zero{]}

4.关于记忆化DP

现在枚举到了 10××× 和 11×××

显然 这两种状态后面的×××状态数是一样的

重点:dp{[}pos{]}{[}limit{]}{[}zero{]}表示前面的数枚举状态确定,后面的数有多少种可能

5.关于DP细节

一般来说我们一开始都memset(dp,-1,sizeof(dp))

如果dp{[}pos{]}{[}limt{]}{[}zero{]}!=-1 return
dp{[}pos{]}{[}limit{]}{[}zero{]};

6.关于初始化:

一开始 limit 是1,表示一开始的数只能选 1\textasciitilde a{[}1{]}

一开始zero 是1,假定表示前面的数全为0

\begin{minted}[fontsize=\footnotesize,breaklines,linenos]{cpp}
#include <bits/stdc++.h>
using namespace std;
#define ll long long
#define mp make_pair
#define pb push_back  // vector函数
#define popb pop_back // vector函数
#define fi first
#define se second
const int N = 20;
// const int M=;
// const int inf=0x3f3f3f3f;     //一般为int赋最大值,不用于memset中
// const ll INF=0x3ffffffffffff; //一般为ll赋最大值,不用于memset中
int T, n, len, a[N], dp[N][2][2];
inline int read()
{
    int x = 0, f = 1;
    char ch = getchar();
    while (ch < '0' || ch > '9')
    {
        if (ch == '-')
            f = -1;
        ch = getchar();
    }
    while (ch >= '0' && ch <= '9')
    {
        x = (x << 1) + (x << 3) + (ch ^ 48);
        ch = getchar();
    }
    return x * f;
}
int dfs(int pos, bool lim, bool zero)
{
    if (pos > len)
        return 1;
    if (dp[pos][lim][zero] != -1)
        return dp[pos][lim][zero];
    int res = 0, num = lim ? a[pos] : 9;
    for (int i = 0; i <= num; i++)
        res += dfs(pos + 1, lim && i == num, zero && i == 0);
    return dp[pos][lim][zero] = res;
}
int solve(int x)
{
    len = 0;
    memset(dp, -1, sizeof(dp));
    for (; x; x /= 10)
        a[++len] = x % 10;
    reverse(a + 1, a + len + 1);
    return dfs(1, 1, 1);
}
int main()
{
    int l = read(), r = read();
    printf("%d\n", solve(r) - solve(l - 1));
    return 0;
}
\end{minted}

\hypertarget{ux72b6ux538bdpux89e3ux54c8ux5bc6ux987fux56deux8defux95eeux9898}{%
\subsection{状压DP解哈密顿回路问题}\label{ux72b6ux538bdpux89e3ux54c8ux5bc6ux987fux56deux8defux95eeux9898}}

\begin{minted}[fontsize=\footnotesize,breaklines,linenos]{cpp}

int dp[(1 << 21)][20];

for (int i = 0; i < (1 << (n + 1)); i++)
            for (int j = 1; j <= n; j++)
                if (dp[i][j])
                    for (int k = 1; k <= n; k++)
                        if (a[j][k] && (((i >> k) & 1) == 0))
                            dp[i | (1 << k)][k] = 1;

\end{minted}

\hypertarget{ux6570ux5b66}{%
\section{4-数学}\label{ux6570ux5b66}}

\begin{center}\rule{0.5\linewidth}{0.5pt}\end{center}

\hypertarget{ux4e09ux5206-1}{%
\subsection{三分}\label{ux4e09ux5206-1}}

\begin{minted}[fontsize=\footnotesize,breaklines,linenos]{cpp}
//寻找cal函数的极小值(极大值则把<改为>)
while(l<r){
    ll lmid=l+(r-l)/3;
    ll rmid=r-(r-l)/3;
    if(cal(lmid)<cal(rmid)) r=rmid-1;//小于,那么l和r会成为较大的值,小于等于则是较小的值
    //例如对于函数(x-40)*(x-41),从整数范围看40和41都能取得极小值
    //而<时,最后l=r=41;<=时,l=r=40
    else l=lmid+1;
}
\end{minted}

\hypertarget{ux5febux901fux5e42}{%
\subsection{快速幂}\label{ux5febux901fux5e42}}

\begin{minted}[fontsize=\footnotesize,breaklines,linenos]{cpp}
int qpow(int a, int n)
{
    int ans = 1;
    while (n)
    {
        if (n & 1)
            ans = ans * a % mod;
        a = a * a % mod;
        n >>= 1;
    }
    return ans;
}
\end{minted}

\hypertarget{exgcd}{%
\subsection{exgcd}\label{exgcd}}

\begin{minted}[fontsize=\footnotesize,breaklines,linenos]{cpp}
int exgcd(int a, int b, int &x, int &y)
{
    if (b == 0)
    {
        x = 1;
        y = 0;
        return a;
    }
    int d = exgcd(b, a % b, x, y), x0 = x, y0 = y;
    x = y0;
    y = x0 - (a / b) * y0;
    return d;
}
\end{minted}

\hypertarget{ux7ebfux6027inv}{%
\subsection{线性inv}\label{ux7ebfux6027inv}}

\begin{minted}[fontsize=\footnotesize,breaklines,linenos]{cpp}
void getinv(int n)
{
    inv[1] = 1;
    for (int i = 2; i <= n; i++)
    {
        inv[i] = mod - ((mod / i) * inv[mod % i]) % mod;
    }
}
\end{minted}

\hypertarget{ux5206ux5757}{%
\subsection{分块}\label{ux5206ux5757}}

\begin{minted}[fontsize=\footnotesize,breaklines,linenos]{cpp}
int ans = 0;
for (int l = 1, r; l <= n; l = r + 1)
{
    r = n / (n / l);
    ans += (r - l + 1) * (n / l);
}
cout << ans << endl;
\end{minted}

\hypertarget{ux6b27ux62c9ux7b5b}{%
\subsection{欧拉筛}\label{ux6b27ux62c9ux7b5b}}

\begin{minted}[fontsize=\footnotesize,breaklines,linenos]{cpp}
int Eular(int n)
{
    int cnt = 0;
    memset(is_prime, true, sizeof(is_prime));
    is_prime[0] = is_prime[1] = false;
    for (int i = 2; i <= n; i++)
    {
        if (is_prime[i])
        {
            prime[++cnt] = i;
        }
        for (int j = 1; j <= cnt && i * prime[j] <= n; j++)
        {
            is_prime[i * prime[j]] = 0;
            if (i % prime[j] == 0)
                break;
        }
    }
    return cnt;
}
\end{minted}

\hypertarget{ux6b27ux62c9ux51fdux6570}{%
\subsection{欧拉函数}\label{ux6b27ux62c9ux51fdux6570}}

\begin{minted}[fontsize=\footnotesize,breaklines,linenos]{cpp}
int Eular(int n)
{
    int cnt = 0;
    memset(is_prime, true, sizeof(is_prime));
    is_prime[0] = is_prime[1] = false;
    for (int i = 2; i <= n; i++)
    {
        if (is_prime[i])
        {
            prime[++cnt] = i;
        }
        for (int j = 1; j <= cnt && i * prime[j] <= n; j++)
        {
            is_prime[i * prime[j]] = 0;
            if (i % prime[j] == 0)
                break;
        }
    }
    return cnt;
}
\end{minted}

\hypertarget{ux7ec4ux5408ux6570}{%
\subsection{组合数}\label{ux7ec4ux5408ux6570}}

\begin{minted}[fontsize=\footnotesize,breaklines,linenos]{cpp}
int fac[N];
int inv[N];
void init(int n)
{
    fac[0] = 1;
    inv[0] = 1;
    inv[1] = 1;
    fac[1] = 1;
    for (int i = 2; i <= 2 * n; i++)
    {
        fac[i] = fac[i - 1] * i % mod;
        inv[i] = (mod - mod / i) * inv[mod % i] % mod;
    }
    for (int i = 1; i <= n; i++)
    {
        inv[i] = inv[i] * inv[i - 1] % mod;
    }
}
int C(int n, int m)
{
    if (m > n || m < 0 || n < 0)
        return 0;
    return fac[n] * inv[m] % mod * inv[n - m] % mod;
}
\end{minted}

\hypertarget{ux6269ux5c55ux6b27ux62c9ux5b9aux7406}{%
\subsection{扩展欧拉定理}\label{ux6269ux5c55ux6b27ux62c9ux5b9aux7406}}

\hypertarget{ux5b9aux4e49}{%
\subsubsection{定义}\label{ux5b9aux4e49}}

\[
a^b \equiv \begin{cases}
  a^{b \bmod \varphi(m)},                &\gcd(a,m) =  1,                   \\
  a^b,                                   &\gcd(a,m)\ne 1, b <   \varphi(m), \\
  a^{(b \bmod \varphi(m)) + \varphi(m)}, &\gcd(a,m)\ne 1, b \ge \varphi(m).
\end{cases} \pmod m
\]

\hypertarget{ux89e3ux91ca}{%
\subsubsection{解释}\label{ux89e3ux91ca}}

读者可能对第二行产生疑问,这一行表达的意思是:如果 \(b < \varphi(m)\)
的话,就不能降幂了。

主要是因为题目中 \(m\) 不会太大,而如果
\(b < \varphi(m)\),自然复杂度是可以接受的。而如果 \(b \ge \varphi(m)\)
的话,复杂度可能就超出预期了,这个时候我们才需要降幂来降低复杂度。

\hypertarget{ux5361ux7279ux5170ux6570}{%
\subsection{卡特兰数}\label{ux5361ux7279ux5170ux6570}}

\begin{minted}[fontsize=\footnotesize,breaklines,linenos]{cpp}
int C(int n, int m)
{
    return fac[n] * qpow(fac[n - m], mod - 2) % mod * qpow(fac[m], mod - 2) % mod;
}
int cat(int n)
{
    return C(2 * n, n) * qpow(n + 1, mod - 2) % mod;
}
\end{minted}

\hypertarget{ux77e9ux9635}{%
\subsection{矩阵}\label{ux77e9ux9635}}

\begin{minted}[fontsize=\footnotesize,breaklines,linenos]{cpp}
class matrix
{
public:
    int x[105][105];
    int sz;
    matrix(int n)
    {
        sz = n;
        for (int i = 1; i <= sz; i++)
        {
            for (int j = 1; j <= sz; j++)
            {
                x[i][j] = 0;
            }
        }
    }
    matrix mul(matrix a, matrix b);
    matrix qpow(matrix a, int n);
    void tra(matrix a);
};

matrix matrix::mul(matrix a, matrix b)
{
    matrix c(a.sz);
    for (int i = 1; i <= a.sz; i++)
        for (int j = 1; j <= a.sz; j++)
            for (int k = 1; k <= a.sz; k++)
                c.x[i][j] = (c.x[i][j] % mod + (a.x[i][k] * b.x[k][j]) % mod) % mod;
    return c;
}
matrix matrix::qpow(matrix a, int n)
{
    matrix res(a.sz);
    for (int i = 1; i <= a.sz; i++)
        res.x[i][i] = 1;
    while (n > 0)
    {
        if (n & 1)
            res = mul(res, a);
        a = mul(a, a);
        n >>= 1;
    }
    return res;
}
void matrix::tra(matrix a)
{
    for (int i = 1; i <= a.sz; i++)
    {
        for (int j = 1; j <= a.sz; j++)
        {
            cout << a.x[i][j] << " ";
        }
        cout << endl;
    }
}
\end{minted}

\hypertarget{ux9ad8ux65afux6d88ux5143}{%
\subsection{高斯消元}\label{ux9ad8ux65afux6d88ux5143}}

求解线性方程组

将各系数合为矩阵,再将其变为上三角矩阵

过程中通常要保证选择的主元绝对值最大以保证精度

n\^{}3

\begin{minted}[fontsize=\footnotesize,breaklines,linenos]{cpp}
const int N = 100;
const double eps = 1e-10;
int n;
double a[N + 1][N + 1], b[N + 1];

void gauss()
{
    int l = 1;
    for (int i = 1; i <= n; i++)
    { // n列
        for (int j = l; j <= n; j++)
        { // 找下面所有行中这一列处绝对值最大的
            if (abs(a[j][i]) > abs(a[l][i]))
            {
                for (int k = i; k <= n; k++)
                {
                    swap(a[l][k], a[j][k]);
                }
                swap(b[l], b[j]);
            }
        }
        if (abs(a[l][i]) < eps)
            continue;
        for (int j = 1; j <= n; j++)
        { // 对所有其他行,更新值
            if (j != l && abs(a[j][i]) > eps)
            {
                double delta = a[j][i] / a[l][i];
                for (int k = i; k <= n; k++)
                {
                    a[j][k] -= a[l][k] * delta;
                }
                b[j] -= b[l] * delta;
            }
        }
        ++l;
    }

    for (int i = l; i <= n; i++)
    { // 假如有剩下的行且b值不为0则无解
        if (abs(b[i]) > eps)
        {
            cout << "无解" << endl;
            return;
        }
    }
    if (l <= n)
    {
        cout << "无穷多解" << endl;
    }
    else
    {
        for (int i = 1; i <= n; i++)
        {
            cout << fixed << setprecision(10) << b[i] / a[i][i] << endl;
        }
    }
}
\end{minted}

\hypertarget{ux7b2cux4e8cux7c7bux65afux7279ux6797ux6570stirling-number}{%
\subsection{第二类斯特林数(Stirling
Number)}\label{ux7b2cux4e8cux7c7bux65afux7279ux6797ux6570stirling-number}}

\textbf{第二类斯特林数}(斯特林子集数)\(\begin{Bmatrix}n\\ k\end{Bmatrix}\),也可记做
\(S(n,k)\),表示将 \(n\) 个两两不同的元素,划分为 \(k​\)
个互不区分的非空子集的方案数。

\hypertarget{ux9012ux63a8ux5f0f}{%
\subsubsection{递推式}\label{ux9012ux63a8ux5f0f}}

\[
\begin{Bmatrix}n\\ k\end{Bmatrix}=\begin{Bmatrix}n-1\\ k-1\end{Bmatrix}+k\begin{Bmatrix}n-1\\ k\end{Bmatrix}
\]

\hypertarget{ux901aux9879ux516cux5f0f}{%
\subsubsection{通项公式}\label{ux901aux9879ux516cux5f0f}}

\[
\begin{Bmatrix}n\\m\end{Bmatrix}=\sum\limits_{i=0}^m\dfrac{(-1)^{m-i}i^n}{i!(m-i)!}
\]

\hypertarget{ux56feux8bba}{%
\section{5-图论}\label{ux56feux8bba}}

\begin{center}\rule{0.5\linewidth}{0.5pt}\end{center}

\hypertarget{dijkstra}{%
\subsection{dijkstra}\label{dijkstra}}

\begin{minted}[fontsize=\footnotesize,breaklines,linenos]{cpp}
#include <bits/stdc++.h>
using namespace std;
#define ll long long
const ll maxn = 10+1e5;
const ll inf = 0x3f3f3f3f;
typedef pair<ll,ll> pll;
    vector<pll> mp[100100];
    ll n,m,s;
    ll dis[100100];
    ll vis[100100];

void dij(ll s){
    for(ll i=1;i<=n;i++){
        dis[i]=inf;
    }
    dis[s]=0;
    priority_queue<pll,vector<pll>,greater<pll> > q;
    q.push({dis[s],s});
    while(!q.empty()){
        ll u=q.top().second;
        q.pop();
        if(vis[u]) continue;
        vis[u]=1;
        for(auto [w,v]:mp[u]){
            if(dis[u]+w<dis[v]){
                dis[v]=dis[u]+w;
                q.push({dis[v],v});
            }
        }
    }
}
int main() {
    cin>>n>>m>>s;
    for(ll i=1;i<=m;i++){
        ll u,v,w;
        cin>>u>>v>>w;
        mp[u].push_back({w,v});
    }
    dij(s);
    for(ll i=1;i<=n;i++)
        cout<<dis[i]<<" ";
    return 0;
}
\end{minted}

\hypertarget{ux5e76ux67e5ux96c6-1}{%
\subsection{并查集}\label{ux5e76ux67e5ux96c6-1}}

\begin{minted}[fontsize=\footnotesize,breaklines,linenos]{cpp}
    ll fa[maxn];
void init(ll n){
    for(ll i=1;i<=n;i++){
        fa[i]=i;
    }
}
ll find(ll x){
    if(fa[x]==x)
        return x;
    fa[x]=find(fa[x]);
    return fa[x];
}
void merge(ll x,ll y){
    if(find(x)==find(y)) return ;
    fa[find(x)]=find(y);
}
\end{minted}

\hypertarget{ux6700ux5c0fux751fux6210ux6811}{%
\subsection{最小生成树}\label{ux6700ux5c0fux751fux6210ux6811}}

\hypertarget{prim-on2}{%
\subsubsection{prim O(n\^{}2)}\label{prim-on2}}

\begin{minted}[fontsize=\footnotesize,breaklines,linenos]{cpp}
    vector<pll> edge[maxn];
    ll dis[maxn];
    ll vis[maxn];
void update(ll u){
    for(auto [v,w]:edge[u]){
        dis[v]=min(dis[v],w);
    }

}
ll prim(ll n){
    for(ll i=1;i<=n;i++){
        vis[i]=0;
        dis[i]=inf;
    }
    vis[1]=1;
    update(1);
    ll ans=0;
    for(ll i=1;i<=n-1;i++){//连接n-1条边
        ll pos,mi=inf;
        for(ll j=1;j<=n;j++){
            if(vis[j]) continue;
            if(dis[j]<mi){
                mi=dis[j];
                pos=j;
            }
        }
        ans+=mi;
        vis[pos]=1;
        update(pos);    
    }
    return ans;
}
\end{minted}

\hypertarget{prim-omlogm-ux5c0fux6839ux5806ux4f18ux5316}{%
\subsubsection{prim O(mlogm)
(小根堆优化)}\label{prim-omlogm-ux5c0fux6839ux5806ux4f18ux5316}}

\begin{minted}[fontsize=\footnotesize,breaklines,linenos]{cpp}
    vector<pll> edge[maxn];
    ll dis[maxn];
    ll vis[maxn];
ll prim(ll n){
    for(ll i=1;i<=n;i++){
        vis[i]=0;
        dis[i]=inf;
    }
    ll ans=0;
    priority_queue<pll,vector<pll>,greater<pll>> pq;
    pq.push({0,1});
    while(!pq.empty()){
        auto [d,u]=pq.top();
        pq.pop();
        if(vis[u]) continue;
        vis[u]=1;
        ans+=d;
        for(auto [v,w]:edge[u]){
            if(!vis[v]&&w<dis[v]){
                dis[v]=w;
                pq.push({dis[v],v});
            }
        }
    }
    return ans;
}
\end{minted}

\hypertarget{kruskal}{%
\subsubsection{kruskal}\label{kruskal}}

\begin{minted}[fontsize=\footnotesize,breaklines,linenos]{cpp}
typedef struct{
    ll u,v,w;
}eg;
    eg e[maxm];
    vector<pll> edge[maxn];
    ll fa[maxn];
bool cmp(eg a,eg b){
    return a.w<b.w;
}
ll find(ll a){
    return (fa[a]==a)?a:(fa[a]=find(fa[a]));
}
void merge(ll a,ll b){
    fa[a]=b;
}
ll kruscal(ll n,ll m){
    for(ll i=1;i<=n;i++){
        fa[i]=i;
    }
    sort(e+1,e+m+1,cmp);
    ll ans=0;
    for(ll i=1;i<=m;i++){
        ll a=find(e[i].u);
        ll b=find(e[i].v);
        if(a!=b){
            merge(a,b);
            ans+=e[i].w;
            n--;
        }
    }
    if(n==1){
        return ans;
    }else{
        return inf;
    }
}
for(ll i=1;i<=m;i++){
    ll u,v,w;
    cin>>u>>v>>w;
    edge[u].push_back({v,w});
    edge[v].push_back({u,w});
    e[i].u=u;e[i].v=v;e[i].w=w;
}
\end{minted}

\hypertarget{lca}{%
\subsection{LCA}\label{lca}}

\begin{minted}[fontsize=\footnotesize,breaklines,linenos]{cpp}
#include <bits/stdc++.h>
using namespace std;
#define IOS ios::sync_with_stdio(false),cin.tie(nullptr), cout.tie(nullptr);
#define ll long long
#define ull unint long long
#define lowbit(i) ((i) & (-i))
#define ls(p) (p << 1)
#define rs(p) (p << 1 | 1)
#define rep(i, a, b) for (ll i = a; i <= b; i++)
#define per(i, a, b) for (ll i = a; i >= b; i--)

typedef pair<ll, ll> pll;
const ll mod = 1e9 + 7;
const ll inf = 0x3f3f3f3f;
const ll maxn = 5e5 + 200;

    ll n, q, root; 
    vector<ll> mp[N];
    ll lg2[maxn];
    ll dep[maxn];
    ll f[maxn][20];
    ll vis[maxn];
void dfs(ll u, ll fa = 0){
    if (vis[u]) return;
    vis[u] = 1;
    dep[u] = dep[fa] + 1;
    f[u][0] = fa;
    for (ll i = 1; i <= lg2[dep[u]]; i++){
        f[u][i] = f[f[u][i - 1]][i - 1];
    }
    for (auto v : mp[u]){
        dfs(v, u);
    }
}
ll lca(ll a, ll b){
    if (dep[a] > dep[b])
        swap(a, b);
    while (dep[a] != dep[b])
        b = f[b][lg2[dep[b] - dep[a]]];
    if (a == b)
        return a;
    for (ll k = lg2[dep[a]]; k >= 0; k--){
        if (f[a][k] != f[b][k]){
            a = f[a][k], b = f[b][k];
        }
    }
    return f[a][0];
}
int main(){
    IOS 
    cin >> n >> q >> root;
    for (ll i = 1; i < n; i++){
        ll u, v;
        cin >> u >> v;
        mp[u].push_back(v);
        mp[v].push_back(u);
    }
    for (ll i = 2; i <= n; i++){
        lg2[i] = lg2[i / 2] + 1;
    }
    dfs(root);
    while (q--){
        ll u, v;
        cin >> u >> v;
        cout << lca(u, v) << endl;
    }
    return 0;
}
\end{minted}

\hypertarget{ux6709ux5411ux56feux5f3aux8054ux901aux5206ux91cf}{%
\subsection{有向图强联通分量}\label{ux6709ux5411ux56feux5f3aux8054ux901aux5206ux91cf}}

有向非强连通图中的\textbf{极大}强连通子图我们称为\textbf{强连通分量}

如果一个有向图不是强连通图但是将\textbf{所有有向边换成无向边}变成了强连通图,那么该图就是\textbf{弱连通图}。

Kosaraju算法

首先对原图 𝐺 进行遍历,记录节点访问\textbf{完}的顺序 𝑑𝑖 , 𝑑𝑖 表示第 𝑖
个访问完的节点编号。

我们选择最晚\textbf{访问完}的节点,对 𝐺
的反向图进行遍历,它能够遍历到的顶点和它组成了一个
SCC,把该过程所遍历到的节点打标记,接下来继续找最晚\textbf{访问完}且未被打上标记的节点进行遍历操作。

\begin{verbatim}
#include <bits/stdc++.h>
#define ll long long
using namespace std;
const int Maxn=1e5+7;
int n,m;
vector<int>e[Maxn],e1[Maxn];
// e 存正向边,e1 存反向边 
bool vis[Maxn];
int d[Maxn],cnt,col[Maxn],cnt1;
void dfs1(int u){
    vis[u]=1;
    for(auto v:e[u]) if(!vis[v]) dfs1(v);
    d[++cnt]=u;
}
vector<int>ans[Maxn];
void dfs2(int u){
    col[u]=cnt1;
    ans[cnt1].push_back(u);
    for(auto v:e1[u]) if(!col[v]) dfs2(v);
}
int main(){
    scanf("%d%d",&n,&m);
    for(int i=1,u,v;i<=m;i++){
        scanf("%d%d",&u,&v);
        e[u].push_back(v);
        e1[v].push_back(u);
    }
    for(int i=1;i<=n;i++) if(!vis[i]) dfs1(i);
    for(int i=n;i;i--) if(!col[d[i]]) ++cnt1,dfs2(d[i]);
    for(int i=1;i<=cnt1;i++) sort(ans[i].begin(),ans[i].end());
    printf("%d\n",cnt1);
    for(int i=1;i<=n;i++){
        if(ans[col[i]].size()){
            for(auto j:ans[col[i]]) printf("%d ",j);
            puts("");
            ans[col[i]].resize(0);
        }
    }
    return 0;
}
\end{verbatim}

Tarjan算法

在 DFS 过程中,我们会遇到如下 4 种边:

\begin{itemize}
\item
  树枝边:DFS 过程中经过的边,即 DFS 搜索树上的边。
\item
  前向边:从祖先节点指向后代节点的非树枝边,我们称为前向边。
\item
  返祖边(后向边):从后代节点指向祖先节点的非树枝边,我们称为返祖边(后向边)。
\item
  横叉边:两端无祖先关系的非树枝边,我们称为横叉边。

  每个强连通分量都是 DFS 树的一颗子树,搜索时,把当前 DFS
  树种未处理的节点加入一个栈,回溯时可以判断栈顶到栈中的节点是否构成一个强连通分量。
\end{itemize}

我们不妨定义 𝑑𝑓𝑛(𝑢) 表示节点 𝑢 在 DFS 中的遍历编号(\textbf{时间戳}),
𝑙𝑜𝑤(𝑢) 表示 𝑢 或 𝑢
的子树能够最多只通过\textbf{一条非树枝边(不包含树边)}回溯的最早的 𝑑𝑓𝑛
值,用一个栈记录经过的节点,那么我们可以得出:

\begin{itemize}
\tightlist
\item
  初始情况有 𝑙𝑜𝑤(𝑢)=𝑑𝑓𝑛(𝑢) 。
\item
  对于边 (𝑢,𝑣) ,如果 𝑢 为 𝑣 的父亲节点,则有
  𝑙𝑜𝑤(𝑢)=min\{𝑙𝑜𝑤(𝑢),𝑙𝑜𝑤(𝑣)\} 。
\item
  对于边 (𝑢,𝑣) 为返祖边或者指向非其他强连通的横叉边,则有
  𝑙𝑜𝑤(𝑢)=min\{𝑙𝑜𝑤(𝑢),𝑑𝑓𝑛(𝑣)\} 。
\end{itemize}

在节点 𝑢 搜索完毕之后,如果 𝑙𝑜𝑤(𝑢)=𝑑𝑓𝑛(𝑢) ,那么说明以 𝑢
为\textbf{根节点}的搜索子树上及栈中在 𝑢
内的元素组成了一个强连通分量,然后删除栈内的这些元素,不断重复该操作直到找到所有的强连通分量。

例 \href{https://www.luogu.com.cn/problem/P3387}{P3387 【模板】缩点 -
洛谷 \textbar{} 计算机科学教育新生态 (luogu.com.cn)}

\begin{verbatim}
#include <bits/stdc++.h>
#define ll long long
using namespace std;
const int Maxn=1e5+7;
int n,m;
vector<int>e[Maxn];
int col[Maxn],dfn[Maxn],low[Maxn],cnt;
int stk[Maxn],top,_cnt;
vector<int>ans[Maxn];
void Tarjan(int u){
    dfn[u]=low[u]=++cnt;
    stk[++top]=u;
    for(auto v:e[u]){
        if(!dfn[v]) Tarjan(v),low[u]=min(low[u],low[v]);
        else if(!col[v]) low[u]=min(low[u],dfn[v]);
    }
    if(low[u]==dfn[u]){
        col[u]=++_cnt;
        ans[_cnt].push_back(u);
        while(stk[top]!=u)  
            ans[_cnt].push_back(stk[top]),col[stk[top--]]=_cnt;
        --top;
    }
}
int main(){
    scanf("%d%d",&n,&m);
    for(int i=1,u,v;i<=m;i++){
        scanf("%d%d",&u,&v);
        e[u].push_back(v);
    }
    for(int i=1;i<=n;i++) if(!dfn[i]) Tarjan(i);
    for(int i=1;i<=_cnt;i++) sort(ans[i].begin(),ans[i].end());
    printf("%d\n",_cnt);
    for(int i=1;i<=n;i++){
        if(ans[col[i]].size()){
            for(auto j:ans[col[i]]) printf("%d ",j);
            puts("");
            ans[col[i]].resize(0);
        }
    }
    return 0;
}
\end{verbatim}

\hypertarget{ux5272ux70b9ux5272ux8fb9}{%
\subsection{割点割边}\label{ux5272ux70b9ux5272ux8fb9}}

\begin{itemize}
\tightlist
\item
  割点:在\textbf{无向图}中,删去该点后使得连通块数增加的结点称为
  \textbf{割点}。
\item
  割边(桥):在\textbf{无向图}中,删去该边后使得连通块数增加的边称为
  \textbf{割边(桥)}。
\end{itemize}

一个图可能会有多个割点或者割边,但是有割点的图不一定存在割边,有割边的图不一定存在割点。

\hypertarget{ux5272ux70b9}{%
\paragraph{割点}\label{ux5272ux70b9}}

和有向图求 SCC 一样,我们会在搜索过程中遇到两种搜索树上的边:

\begin{itemize}
\tightlist
\item
  树枝边:DFS 过程中经过的边,即 DFS 搜索树上的边。
\item
  返祖边(后向边):从后代节点指向祖先节点的非树枝边,我们称为返祖边(后向边)。
\end{itemize}

不包含横叉边和前向边,因为这是\textbf{无向图}。

我们不妨定义 𝑑𝑓𝑛(𝑢) 表示节点 𝑢 在 DFS 中的遍历编号(\textbf{时间戳}),
𝑙𝑜𝑤(𝑢) 表示 𝑢 或 𝑢
的子树能够最多只通过\textbf{一条非树枝边(不包含树边)}回溯的最早的 𝑑𝑓𝑛
值,那么我们可以得出:

\begin{itemize}
\tightlist
\item
  初始情况有 𝑙𝑜𝑤(𝑢)=𝑑𝑓𝑛(𝑢) 。
\item
  对于边 (𝑢,𝑣) ,如果 𝑣 没有被搜索到,那么这条边就是树枝边,则有
  𝑙𝑜𝑤(𝑢)=min\{𝑙𝑜𝑤(𝑢),𝑙𝑜𝑤(𝑣)\} 。
\item
  对于边 (𝑢,𝑣) ,如果 𝑣 被搜索到了,那么这条边就是返祖边,则有
  𝑙𝑜𝑤(𝑢)=min\{𝑙𝑜𝑤(𝑢),𝑑𝑓𝑛(𝑣)\} 。
\end{itemize}

对于一个节点 𝑢 ,它的子节点 𝑣 ,若 𝑙𝑜𝑤(𝑣)≥𝑑𝑓𝑛(𝑢) ,说明 𝑣
无法通过它子树的节点到达 𝑑𝑓𝑛 更小的节点,也就是 𝑣 无法不经过点 𝑢 到达比
𝑢 的 𝑑𝑓𝑛 值更小的节点,显然 𝑢 是一个割点,反之 𝑢 就不是一个割点。

但是对于根节点 𝑢 ,它的 𝑑𝑓𝑛
值一定是整个序列的最小值,因此上述方法不管用,如果它存在两个及以上的子节点,那么把
𝑢 删除后绝对会把 𝑢 子节点的子树分割开来,此时 𝑢 是割点。

\begin{minted}[fontsize=\footnotesize,breaklines,linenos]{cpp}
#include <bits/stdc++.h>
#define ll long long
using namespace std;
const ll Maxn=1e5+7;
struct edge1{
    ll v,Next;
}Edge[Maxn<<1];
ll n,m,tot,low[Maxn],dfn[Maxn],cnt,root,ans,head[Maxn];
bool flg[Maxn];
inline void add(ll u,ll v){
    Edge[++tot]=(edge1){v,head[u]},head[u]=tot;
}
void tarjan(ll u){
    dfn[u]=low[u]=++cnt;
    ll ch=0;
    for(ll i=head[u];i;i=Edge[i].Next){
        ll v=Edge[i].v;
        if(!dfn[v]){
            ch++;
            tarjan(v);
            low[u]=min(low[u],low[v]);
            if(low[v]>=dfn[u]&&u!=root) flg[u]=1;
            
        }
        else low[u]=min(low[u],dfn[v]);
    }
    if(ch>=2&&root==u) flg[u]=1;
}
int main(){
    scanf("%lld%lld",&n,&m);
    for(ll i=1,u,v;i<=m;i++) 
        scanf("%lld%lld",&u,&v),add(u,v),add(v,u); 
    for(ll i=1;i<=n;i++)
        if(!dfn[i])
            root=i,tarjan(i);
    for(ll i=1;i<=n;i++)
        if(flg[i])
            ++ans;
    printf("%lld\n",ans);
    for(ll i=1;i<=n;i++)
        if(flg[i])
            printf("%lld ",i);
    return 0;
}
\end{minted}

\hypertarget{ux5272ux8fb9}{%
\paragraph{割边}\label{ux5272ux8fb9}}

割边的求法和割点的求法类似,我们继续使用相同定义的 𝑙𝑜𝑤 和 𝑑𝑓𝑛 。

很显然,树枝边使得整个图连通,而非树边删除后并不影响图的连通性,因此\textbf{割边一定是树枝边}。

假设当前节点为 𝑢 ,它有子节点 𝑣 ,那么边 (𝑢,𝑣) 为割边时当且仅当
𝑙𝑜𝑤(𝑣)\textgreater 𝑑𝑓𝑛(𝑢) ,只要 𝑣 的节点能通过非树边来到比 𝑢 的 𝑑𝑓𝑛
更小的节点,那么 (𝑢,𝑣)
就不是割边,不取等号的原因是它是一条边而非一个节点。

现在唯一的问题就是判断一条边是否为非树边,我们不能直接将 𝑣→𝑢 ( 𝑢 是 𝑣
的父亲)识别成非树边,这样可能将树边也识别成非树边。

tarjan

\begin{minted}[fontsize=\footnotesize,breaklines,linenos]{cpp}
#include<bits/stdc++.h>
#define int long long
const int N=1e6+1;
using namespace std;
struct fy
{
    int v,next;
}edge[N];
struct fy_
{
    int from,to;
}E[N];
int dfn[N],low[N],n,m,x,y,idx,head[N],res,cnt,IDX;
bool g[N];
inline void add(int x,int y)
{
    edge[++cnt].v=y,edge[cnt].next=head[x],head[x]=cnt;
}
inline void dfs(int u,int fa)
{
    dfn[u]=low[u]=++idx;
    for(int i=head[u];i;i=edge[i].next)
    {
        int v=edge[i].v;
        if(!dfn[v])
        {
            dfs(v,u);
            low[u]=min(low[u],low[v]);
            if(low[v]>dfn[u])
                E[++IDX].from=min(u,v),E[IDX].to=max(u,v);
        }
        else if(v!=fa&&dfn[v]<dfn[u])
            low[u]=min(low[u],dfn[v]);
    }
}
signed main()
{
    scanf("%lld%lld",&n,&m);
    for(int i=1;i<=m;i++)
    {
        scanf("%lld%lld",&x,&y);
        add(x,y),add(y,x);
    }
    dfs(1,-1);
    for(int i=1;i<=IDX;i++)
        printf("%lld %lld\n",E[i].from,E[i].to);
}
\end{minted}

树上差分

对于每一条非树边,在其树上深度较小的点处打上 \texttt{-1}
标记,在其树上深度较大的点处打上 \texttt{+1}
标记。然后求出每个点的子树内部的标记之和。对于一个点u,其子树内部的标记之和等于覆盖了
u和u的父亲之间的树边的非树边数量。若这个值非0,则u和u的父亲之间的树边不是桥,否则是桥。

\hypertarget{ux53ccux8054ux901aux5206ux91cf}{%
\subsection{双联通分量}\label{ux53ccux8054ux901aux5206ux91cf}}

双连通分量分为\textbf{点双连通分量}和\textbf{边双连通分量}。

边双连通具有传递性,点双连通 \textbf{不} 具有传递性

\begin{itemize}
\item
  点双连通图:\textbf{不存在割点的无向连通图}我们称为\textbf{点双联通图}。
\item
  边双连通图:\textbf{不存在割边的无向连通图}我们称为\textbf{边双联通图}。
\item
  点双连通分量(点双):一张图的极大点双连通子图称为\textbf{点双连通分量(V-BCC)。}
\item
  边双连通分量(边双):一张图的极大边双连通子图称为\textbf{边双连通分量(E-BCC)}。
\item
  点双连通:若两点 𝑢,𝑣 在同一个点双连通分量内,那么我们称 𝑢,𝑣 点双连通。
\item
  边双联通:若两点 𝑢,𝑣 在同一个边双连通分量内,那么我们称 𝑢,𝑣 边双连通。

  在一张无向图中,如果 (𝑢,𝑣) 直接相连,那么 𝑢,𝑣
  点双连通,但不一定边双连通。

  同时,双连通分量还有一个很好的性质,我们将强连通分量缩点后可以得到一个
  \textbf{DAG},但是双连通分量缩点之后可以得到\textbf{一棵树},也就是我们在进阶中要讲的\textbf{圆方树}。
\end{itemize}

\hypertarget{ux8fb9ux53ccux8fdeux901aux5206ux91cf}{%
\subsubsection{边双连通分量}\label{ux8fb9ux53ccux8fdeux901aux5206ux91cf}}

边双连通分量事实上很好求。

我们只需将该无向图的所有割边删去,整张图就会分裂成\textbf{割边数量+1}个联通块,这些联通块内不存在割边,也就是该图分裂成了若干个边双连通分量。

\begin{minted}[fontsize=\footnotesize,breaklines,linenos]{cpp}
#include <bits/stdc++.h>
#define ll long long
using namespace std;
const ll Maxn=4e6+7;
struct edge1{
    ll u,v,Next;
}Edge[Maxn<<1];
ll n,m,tot=1,f[Maxn];
ll low[Maxn],dfn[Maxn],cnt,head[Maxn];
ll cl;
vector<ll>ans[Maxn];
bool flg[Maxn],vis[Maxn];
inline void add(ll u,ll v){
    Edge[++tot]=(edge1){u,v,head[u]},head[u]=tot;
}
void tarjan(ll u){
    dfn[u]=low[u]=++cnt;
    for(ll i=head[u];i;i=Edge[i].Next){
        if(i==(f[u]^1)) continue;
        ll v=Edge[i].v;
        if(!dfn[v]){
            f[v]=i;
            tarjan(v);
            low[u]=min(low[u],low[v]);
            if(low[v]>dfn[u]) flg[i]=flg[i^1]=1;
        }
        else low[u]=min(low[u],dfn[v]);
    }
}
void DFS(ll u){
    vis[u]=1;
    ans[cl].push_back(u);
    for(ll i=head[u];i;i=Edge[i].Next){
        if(!flg[i]&&!vis[Edge[i].v]){
            DFS(Edge[i].v);
        }
    }
}
int main(){
    scanf("%lld%lld",&n,&m);
    for(ll i=1,u,v;i<=m;i++) 
        scanf("%lld%lld",&u,&v),add(u,v),add(v,u); 
    for(ll i=1;i<=n;i++) if(!dfn[i]) tarjan(i);
    for(ll i=1;i<=n;i++) if(!vis[i]) ++cl,DFS(i);
    printf("%lld\n",cl);
    for(ll i=1;i<=cl;i++,puts("")){
        printf("%lld ",ans[i].size());
        for(auto j:ans[i]) printf("%lld ",j);
    }
    return 0;
}
\end{minted}

\hypertarget{ux70b9ux53ccux8fdeux901aux5206ux91cf}{%
\subsubsection{点双连通分量}\label{ux70b9ux53ccux8fdeux901aux5206ux91cf}}

对于一个点双,它在 DFS 搜索树中 𝑑𝑓𝑛\emph{d\textbf{f}n}
值最小的点一定是割点或者树根。

当这个点是割点时,它所属的点双必定不可以向它的父亲方向包括更多点,因为一旦回溯,它就成为了新的子图的一个割点,不是点双。所以它应该归到其中一个或多个子树里的点双中。

当这个点是树根时,它的 𝑑𝑓𝑛\emph{d\textbf{f}n}
值是整棵树里最小的。它若有两个以上子树,那么它是一个割点;它若只有一个子树,它一定属于它的直系儿子的点双,因为包括它;它若是一个独立点,视作一个单独的点双。

换句话说,一个点双一定在这两类点的子树中。

我们用栈维护点,当遇到这两类点时,将子树内目前不属于其它点双的非割点或在子树中的割点归到一个新的点双。注意这个点可能还是与其它点双的公共点,所以不能将其出栈。

\begin{minted}[fontsize=\footnotesize,breaklines,linenos]{cpp}
#include <bits/stdc++.h>
using namespace std;
const int N = 5e5 + 5, M = 4e6 + 5;
int cnt = 1, fir[N], nxt[M], to[M];
int s[M], top, bcc, low[N], dfn[N], idx, n, m;
vector<int> ans[N];
inline void tarjan(int u, int fa) {
    int son = 0;
    low[u] = dfn[u] = ++idx;
    s[++top] = u;
    for(int i = fir[u]; i; i = nxt[i]) {
        int v = to[i];
        if(!dfn[v]) {
            son++;
            tarjan(v, u);
            low[u] = min(low[u], low[v]);
            if(low[v] >= dfn[u]) {
                bcc++;
                while(s[top + 1] != v) ans[bcc].push_back(s[top--]);//将子树出栈
                ans[bcc].push_back(u);//把割点/树根也丢到点双里
            }
        } else if(v != fa) low[u] = min(low[u], dfn[v]);
    }
    if(fa == 0 && son == 0) ans[++bcc].push_back(u);//特判独立点
}
inline void add(int u, int v) {
    to[++cnt] = v;
    nxt[cnt] = fir[u];
    fir[u] = cnt;
}
int main() {
    scanf("%d%d", &n, &m);
    for(int i = 1; i <= m; i++) {
        int u, v;
        scanf("%d%d", &u, &v);
        add(u, v), add(v, u);
    }
    for(int i = 1; i <= n; i++) {
        if(dfn[i]) continue;
        top = 0;
        tarjan(i, 0);
    }
    printf("%d\n", bcc);
    for(int i = 1; i <= bcc; i++) {
        printf("%d ", ans[i].size());
        for(int j : ans[i]) printf("%d ", j);
        printf("\n");
    }
    return 0;
}
\end{minted}

\end{document}
